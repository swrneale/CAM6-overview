\begin{abstract}

The NCAR Community Atmosphere Model, version 6 (CAM6), represents a significant advance in model capability, both in the representation of physical processes and simulation performance. Existing dry and moist turbulence parameterizations are now represented by the Cloud Layers Unified By Binormals (CLUBB) moist turbulence scheme. Its primary advance is a continuous representation of many physical processes previously described in conceptually separate parameterizations. Further improvements have been made to microphysical, aerosol and orographic impacts on momentum budgets.

Simulated model performance surpasses that of CAM5 for many mean climate features including precipitation, humidity and short-wave cloud forcings, each of which are integral quantities setting climate sensitivity of CESM2. While northern winter circulation is improved, summertime climate experiences a marked degradation due to a poleward excursion of the Asia-Pacific jet. This is primarily the result of upper tropospheric wind biases whose barotropic signature translates to surface stress biases in the mid-latitudes. We demonstrate that although CAM6 improves skill, it can often reflect a recovery of bias increases seen between CAM4 and CAM5.

I terms of the model physical parameterizations there is a changed balance of contributions in CAM6. Regionally CLUBB thermodynamic tendencies match the sum of the equivalent processes that are represented separately in CAM5. The caveat to this is that there is significant compensation of weakened deep convection in the CLUBB response. Furthermore, column conditions over key climate regions differ markedly. In particular, the magnitude of cloud fraction and cloud water diverges amongst model versions, with obvious implications for climate sensitivity.

Sensitivity experiments reveal that the largest precipitation changes in the South Pacific Convergence Zone (SPCZ) and the South Asian Monsoon, are predominantly attributable to the changes in the microphysics scheme and in the transition to CLUBB. Cloud forcing changes between CAM5 and CAM6 reflect a complex superposition of parameterization changes. However, tuning parameter modifications, often for the purpose of coupled climate tuning, also play a significant role in these same regions. Significant performance degradation is seen in the large-scale flow. Increased tropospheric biases reflect an erroneous poleward shift in the westerly jets, and this can be attributed to the shift to CLUBB and improvements to the microphysics parameterization. More complicated causes emerge in Summertime where tuning parameter changes made during model development are equally as important.



%The NCAR Community Atmosphere Model, version 6 (CAM6) is the atmospheric component of the Community Earth System Model, version 2 (CESM2). It represents a significant advance in capability, both in the representation of physical processes and simulation skill. Existing dry and moist turbulence parameterizations are now represented by the Cloud Layers Unified By Binormals (CLUBB) moist turbulence scheme. Its primary advance is a continuous representation of many physical processes previously described in conceptually separate parameterizations. 

%Microphysical processes are represented by a version of the Morrison-Gettelman (MG2) microphysics that now predicts mass and number concentrations of rain and snow. Aerosol processes include an additional mode as part of the Modal Aerosol Scheme (MAM4), more accurately representing black carbon chemical processes.  A new surface drag parameterization includes stress tendencies that are distributed above the lowest model level. Modifications to the gravity wave scheme enable anisotropic sub-grid orographic characteristics to determine vertically propagating gravity wave emission, in response to wind direction

%The performance in AMIP simulations surpasses that of CAM5 for many mean climate features including precipitation, humidity and critically short-wave cloud forcings, each of which are integral quantities setting climate sensitivity of CESM2. While northern winter circulation is improved, summertime climate experiences a marked degradation due to a poleward excursion of the Asia-Pacific jet. This also reflects the fact that the largest degradation in CAM6 is the result of upper tropospheric wind biases whose barotropic signature translates to surface stress biases in the mid-latitudes. Although certain features are improved in CAM6, it is noteworthy they reflect a recovery of bias increases seen between CAM4 and CAM5.

%Regionally CLUBB thermodynamic tendencies match the sum of the equivalent processes that are represented separately in CAM5. The caveat to this is that there is clearly compensation of weakened deep convection by CLUBB. Vertical column conditions over key climate regions differ markedly. In particular, the magnitude of cloud fraction and cloud water diverges significantly amongst model versions, with obvious implications for climate sensitivity.

%Sensitivity experiments reveal that the largest precipitation changes in the South Pacific Convergence Zone (SPCZ) and the South Asian Monsoon, are predominantly attributable to the changes in the microphysics scheme and in the transition to CLUBB. Cloud forcing changes between CAM5 and CAM6 reflect a complex superposition of parameterization changes. However, tuning parameter modifications, often for the purpose of coupled climate tuning, also play a significant role in these same regions. Significant performance degradation is seen in the large-scale flow. Increased tropospheric biases reflect an erroneous poleward shift in the westerly jets, and this can be attributed to the shift to CLUBB and improvements to the microphysics parameterization. More complicated causes emerge in Summertime where tuning parameter changes made during model development are equally as important.







\end{abstract}
\pagebreak