\begin{abstract}

The NCAR/DOE Community Atmosphere Model, version 6 (CAM6) is the atmosphere component of the Community Earth System Model, version 2 (CESM2). It is a significant advance from previous model versions, in the representation physical process and simulation performance. With the exception of the deep convection parameterization all macroscale cloud processes and dry turbulence parameterizations have been replaced by the Cloud Layers Unified By Binormals (CLUBB) unified cloud and turbulence scheme. It simulates continuously multiple turbulence regimes, including clouds, dry convective and stably stratified meteorology. Its major advantage is a continuous representations of many physical processes that were previous represented by conceptually separate parameterizations. Microphysical processed are now represented by an updated version of the Morrison-Gettelman (MG2) microphysics that includes predicted mass and number concentrations of rain and snow. Aerosol processes include an additional mode as part of the Modal Aerosol Scheme (MAM4) the additional mode is able to more accurately represent black carbon chemical processes. The Zhang-McFarlane (ZM) deep convection scheme exists for the most as in CAM5, with minor modifications for plume stability sensitivity. Finally surface drag parameterization and gravity wave emission schemes have been significantly updated to include stress tendencies that now impact abobe the lowest model level

The performance in AMIP simulations surpasses that of CAM5 for many mean climate features including precipitation, humidity and critically short-wave cloud forcings that are integral to the increased climate sensitivity of CESM2. Regionally CLUBB thermodynamic tendencies match the sum of the similar processes that are represented separately in CAM5. The caveat to this is that there is clearly compensation for the deep convection, which is reduced and consequently results in a convective precipitation fraction lower than CAM5, which is considered more realistic. 

The hydrostatic Finite Volume dynamical core is retained from CAM5, and will be used for contributions to CMIP6. 

Atmosphere-only (AMIP-type) simulations demonstrate significantly improved skill compared to CAM5.

\end{abstract}
\pagebreak