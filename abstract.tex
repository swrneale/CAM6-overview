\begin{abstract}

The NCAR/DOE Community Atmosphere Model, version 6 (CAM6) is the atmosphere component of the Community Earth System Model, version 2 (CESM2). It is a significant advance from previous model versions, in the representation of physical procesess and simulation performance. With the exception of the deep convection parameterization, all macroscale cloud processes and dry turbulence parameterizations have been replaced by the Cloud Layers Unified By Binormals (CLUBB) unified moist turbulence scheme. It simulates co-occurring turbulence regimes, including clouds, dry convective and stably stratified regimes. Its major advantage is a continuous representation of many physical processes that were previously represented in conceptually separate parameterizations. 

Microphysical processes are now represented by an updated version of the Morrison-Gettelman (MG2) microphysics that includes predicted mass and number concentrations of rain and snow. Aerosol processes include an additional mode as part of the Modal Aerosol Scheme (MAM4) the additional mode is able to more accurately represent black carbon chemical processes.  New surface drag parameterization and gravity wave schemes have been significantly updated to include stress tendencies that now impact above the lowest model level, and anisotropic sub-grid orographic for wave emission.

The performance in AMIP simulations surpasses that of CAM5 for many mean climate features including precipitation, humidity and critically short-wave cloud forcings that are integral to the increased climate sensitivity of CESM2. Regionally CLUBB thermodynamic tendencies match the sum of the equivalent processes that are represented separately in CAM5. The caveat to this is that there is clearly compensation for the deep convection, which is reduced and consequently results in a convective precipitation fraction lower than CAM5, which is considered more realistic. The performance compared to CAM5 is marginal, but in the coupled model it's significant and results from MAM and DJF improvements of poor performance in CESM1, especially as it relates to tropical Pacific surface stresses and rainfall.



An analysis of sensitivity experiments reveals that precipitation skill improvements are, to a certain extent, attributable to increases in deep convection sensitivity during Winter. However more complicated causes emerge in Summertime where tuning parameters changed during model development would seem to be equally as important. In actual fact, interim tuning parameters used in coupled model development deliver a better climate than the release version of CAM6.

Significant performance degradation is, however, seen in the large-scale flow. Increased tropospheric biases reflect an erroneous poleward shift in the westerly jets, and this can be attributed to the shift to CLUBB and improvements to the microphysics parameterization.

\end{abstract}
\pagebreak