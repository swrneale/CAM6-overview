\section{Mean Climate Skill}
\label{sec:climate}

There have been significant simulation improvements in CAM6 relative to CAM5. Many of these can be attributed to changes in individual physical process, but a number have occurred due to the calibration or 'tuning' carried out in the context of a fully coupled version of CESM2. The Normalized Mean Standard Error (NMSE) shown in Fig. \ref{f_CAM_CESM_NMSE_DJF_JJA} summarizes error statistics, including bias and phase errors, for the northern hemisphere seasonal 200-mb height fields. The legacy of models back to CAM3 \citep{Collins2006} generally shows December/January/February (DJF) errors to be smaller when SSTs are prescribed (AMIP), versus when an active ocean model is used. CAM6 departs from this behavior as the CESM2 error is of comparable magnitude, reflecting the focus shift in the model development cycle, between AMIP and coupled. Although the CESM2 unconditional bias is reduced by a factor of four from CESM1, changes over time have been monotonic. A significant degradation from both CCSM3 \citep{Collins2006} and CCSM4 \citep{Gent2011} to CESM1 exists, and reflects a cool late 20th century climate, as well as cyclonic biases over the ocean basins. 

Errors in DJF are generally determined by variations in the oceanic storm track climate, a well simulated phenomenon in climate models. In June/July/August (JJA), however, errors are characterized by large scale monsoonal circulations and rely on more poorly constrained land-atmosphere coupling \citep{Dirmeyer2006}, hence the higher overall NMSE. Although CESM2 errors indicate it is the most skillful coupled model, the CAM6 AMIP simulations do not perform nearly as well as even CAM4. Given the non-negligible SST biases in CESM2 \citep{Danabasoglu2020}, one could infer that CESM2 performs better for the wrong reasons, which will become apparent in later analyses. Errors for all model configurations easily exceed differences among different analyses products by a factor of ten in DJF and a factor of five in JJA. The trend in NCAR model development indicates that CCSM4 may be a more skillful model than CESM1, and for this reason we also consider the performance of CCSM4 in what follows.

A broader summary of the change in model skill is shown through Taylor Diagrams \citep[Fig.\ref{f_TAYLOR_CAM_CESM},][]{Taylor2001} for both AMIP and coupled versions using CESM. Systematic improvements in the most challenging climate fields (such as precipitation and cloud forcing fields) are only marginally apparent in AMIP simulations. Correlation of precipitation fields improve by almost 0.1 between CAM5 and CAM6, but this comes at the cost of degradation in the seasonal, temporal and spatial variance (standard deviation) over land. It is recognised that different versions of the community land model (CLM) can also influence this pattern  The coupled simulations contrast markedly with AMIP, since the less skillful fields (including Pacific zonal surface stress) are systematically improved between CAM4 and CAM6. Cloud radiation fields improve markedly, the regional changes of which are highlighted below. While the scaled RMSE improves to 0.82 in CESM2, the mean bias has improved only after a significant degradation in CESM1 and CAM5 (similar to above) and that was primarily related to more poorly simulated long wave cloud forcing and ocean precipitation in this model version.

The regional precipitation distribution is significantly improved in CAM6; decreasing RMSE by more than 10$\%$, but Indo-Pacific biases still continue to dominate (Fig. \ref{f_PRECT_2D_CAM456}). Over the Western Indian Ocean and Arabia wet biases have been mostly illuminated, decreasing from in excess of 3 $mm/day$ and corresponding to an excess of 100$\%$ of the observed values. Seasonally, this reduction occurs in the pre-monsoon period, and is reflective of CLUBB effectively capping cloud depth within the southerly and south-westerly flow region REF??. The Inter Tropical Convergence Zone (ITCZ) remains a persistent problem in CMIP models \citep{Tian2020} and CAM is no exception, with persistence in ITCZ errors through previous model versions. The Pacific ITCZ bias is virtually illuminated in CAM6, the source of which will be discussed in section \ref{sec:sensitivities}. In contrast to this is the Atlantic ITCZ where precipitation amounts are now lower than 25\% of observed.

In the seasonal zonal mean distribution (Fig. \ref{f_PRECT_1D_DJF_CAM456}), there are clear improvements seen in DJF and most obviously related to a 50\% reduction from the East Pacific ITCZ. This is not reflected in coupled CESM2, where wet biases reside mostly to the South of the equator. Similarly in CESM1 and CCSM4, where AMIP biases tend to be small, the coupled bias are largest. In JJA (Fig. \ref{f_PRECT_1D_JJA_CAM456}) the relationship between the East Pacific ITCZ biases in AMIP and coupled models changes compared to DJF. Coupled CESM2 wet biases are larger than 2 $mm/day$ where CAM6 is dryer than observed. Contrasting coupled versus AMIP precipitation biases serves to emphasize the locking of the precipitation to the prescribed SSTs in all model versions, but with greater variations seen in the bias pattern of coupled configurations.

Although the ITCZ bias is largely remedied in DJF it does not accompany decreased biases in the upper tropospheric flow. The 200-mb rotational and divergent circulation (Fig. \ref{f_PSICHI_2D_DJF_CAM456_diff}), which are markers of the large-scale response to tropical convective heating, imply an anomalous increase in the CAM6 convective outflow compared to CAM5, when biases were minimal. In the DJF season a drying in the West Pacific reaching in excess of 3 $mm/day$ (not shown) enables the divergent center of action to move Eastward. In response the rotational flow reflects an anomalous \citep{Gill1980} type response, amplifying the climatological large-scale response. This biased response mirrors more closely that in CAM4 than in CAM5, where the precipitation bias distributions are similar. An improvement in the distribution of precipitation coupled with a degradation to the upper tropospheric is indicative of biases in the vertical distribution of tropical heating that ultimately drives the upper tropospheric flow. This could be a consequence of the ZM scheme's greater stability constraints in CAM6 leading to more bottom heaving convective heating distributions, even though the same changes have contributed to improved sub-seasonal variability in the model \citep{Danabasoglu2020,Meehl2020}.

Although the precipitation improvements in CAM6 are more regional in nature, the simulations of cloud radiative forcings reveals clear cut improvements more global in nature. For short wave cloud forcing (SWCF) the changes in CAM6 represent systematic improvements almost everywhere. In particular the string cooling biases due to low cloud properties that were reduced over the tropical oceans in CAM5 are reduced further over tropical land regions. At higher southern latitudes ice edge deficiencies which strengthened slightly in CAM5 have been reduced. Degradation in the forcing is mostly limited to reduced cloudiness over sub-tropical stratocumulus regions and the complex makeup of this bias is shown in \ref{sec:sensitivities}. For long wave cloud forcing (LWCF, Fig. \ref{f_LWCF_1D_ANN_CAM456}), there are also systematic changes across model versions. In CAM5 where liquid and ice number concentrations are predicted in MG1 a 10 Wm$^{-2}$ reduction in LWCF was seen across the tropics compared to CAM4. CAM6 with MG2 is able to recover by around 50\% compared to observed, but in all models no appreciable improvements in the 5-10 Wm$^{-2}$ too weak forcing at higher latitudes is seen. 

The zonally averaged circulation features reveal a number of key changes to the climate in CAM6. The Mean Meridional Circulation (Fig. \ref{f_MMC_2D_ANN_CAM456}) somewhat paradoxically shows that CAM4 and CAM5 had a weakened Hadley circulation even in the presence of an overly strong East Pacific ITCZ. Consequently CAM6 weakens the MMC with a more skillful ITCZ, hinting at an overly efficient coupling between ascent and convective heating in the model. 

The most significant improvement in the temperature field in CAM6 is the large reduction of the 6-8K cold pole problem seen over the Arctic, in particular, in both CAM4 and CAM5 (Fig. \ref{f_T_2D_ANN_CAM456}). In the tropics the colder tropopause feature is likely due to more moisture sensitive deep convection, but free tropospheric cold biases are largely unchanged. 

q??

Perhaps the most concerning degradation in climate are features of the dynamical simulation (Fig. \ref{f_U_2D_ANN_CAM456}). CAM5 saw significant improvements in the near surface zonal flow compared to CAM4, but a similar distribution of excess tropical easterlies and mid-latitude westerlies return in CAM6. Perhaps, more impactful the zonal jet biases between 100 mb and 300 mb are worsened at nearly every latitude, changing biases from 1-2 ms$^{-1}$ generally to closer to 4 ms$^{-1}$.