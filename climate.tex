\section{Global Diagnostics}
\label{sec:climate}
What is good?

Figs
-Taylor diagram
-Global averages of PRECT (to see if there are too strong a peak in DJF/JJA)
-SKill scores
-Zonally averaged PRECT,PS

Improvements yes, but tuning makes a big difference from CAM6p to CAM for example.

There have been significant improvements in CAM6 relative to CAM5 for a number of climate features. Many of the improvements can be attributed to changes in individual physical process, but it is fair to see that many improvements occurred due to the calibration or 'tuning' process carried out in the context of fully coupled simulations and required in order to provide a model for CMIP6 experiments. Here we are showing AMIP simulations, but it is important to state that the calibration process was predominantly performed as part of producing a credible fully coupled climate. 
ITCZ is not reduced as much in CESM2

The regional precipitation distribution over the global has seen significant improvement in CAM6. In Figure \ref{f_PRECT_2D_CAM456} Indo-Pacific biases dominate, but so do improvements in the same region. In the Western Indian Ocean reductions in the wet biases are reduced, which continues the trend since CAM4. Seasonally this occurs in the pre-monsoon period, and is reflective of CLUBB being able to cap precipitation depth within the southerly and south-westerly flow region.

The Inter Tropical Convergence Zone (ITCZ) is a perpetual problem for climate models and CAM is no exception, particularly in the DJF season. Fig \ref{f_zm_precip} shows persistence in ITCZ errors through previous model versions. The ITCZ did see somewhat of degradation in CAM4 and CAM5, but the regions of peak precipitation have seen a significant improvement in CAM6 by around 50\%. 

Although the ITCZ bias is remedied it does introduce biases in the upper tropospheric flow. The reduction in the ITCZ strength in the central Pacific and presumably, ascent and upper tropical divergent flow displaces the divergent center of action to the west. The rotational and divergent circulation (REF Fig) are closely coupled at large-scales. Since ITCZ not reduced as much in CESM2, upper tropospheric circulation is displaced.

