\section{Introduction}

This article documents the development of the Community Atmosphere Model, version 6 (CAM6) configuration and features Atmospheric Model Intercomparison Project \citep[AMIP, ][]{}. CAM6 forms the atmospheric component of the Community Earth System Model, version 2 (CESM2, \citep) which will be used for extensive contributions to the CMIP6 project and users across the earth system science community. CAM6 representations a significant enhancement in the the representation of atmospheric processes. Following the development of CAM5 used in CESM1 and CMIP5, decisions were taken to develop capability targeting incomplete process representations, and poor representations of processes key to climate sensitivity. The representation of clouds and cloud processes continue to be at the forefront of this development. One particular problem that was addressed is the discontinuity between different parameterizations of closely related and interacting processes, namely shallow convection, boundary layer processes and gird-scale condensation. Unifying these processes has been achieved through the implementation of the Cloud Layers Unified By Binormals (CLUBB) scheme. Microphysical processes.