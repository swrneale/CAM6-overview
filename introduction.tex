\section{Introduction}
\label{sec:intro}

The National Center for Atmospheric Research (NCAR) Community Atmosphere Model, version 6 (CAM6) forms the atmospheric component of the Community Earth System Model, version 2 (CESM2). CESM is used for extensive contributions to the current Coupled Model Intercomparison Project (CMIP6). We document the scientific configurations of CAM6 and analysis changes of the climate features in Atmospheric Model Intercomparison Project \citep[AMIP,][]{Gates1992a} simulations.

CAM6 represents a significant enhancement in the representation of atmospheric processes compared to CAM5  \citep[][]{Neale2012}. Following the development of CAM5 used in CESM1 and CMIP5, we develop capabilities that target incomplete process representations, and in particular processes key to climate sensitivity and cloud aerosol processes. The representation of clouds and cloud processes continue to be at the forefront of this development. One particular problem that was addressed is the discontinuity between different parameterizations of closely related and interacting processes, namely shallow convection, boundary layer processes and gird-scale condensation. Unifying these processes has been achieved through the implementation of the Cloud Layers Unified By Binormals (CLUBB, REF) scheme. Microphysics, surface stress, vertically propagating gravity waves and aerosol processes parameterizations further. The combination of these processes have combined to produce climate forcing, and feedbacks significantly different to CAM5, with the net result that climate sensitivity is significantly higher in CESM2 coupled simulations REF CAM6 clim. 




Unlike previous model development CAM5 and CESM1, CAM6, CESM2 and the high-rop version of CAM6, the Whole Atmosphere Community Climate Model were development mostly in unison. This mean any 

With this multi-model development paradigm in mind, the tuning process is subject to more constraints than previously in the choice of parameter settings used to deliver CESM2. These arise not just to produce a good atmosphere simulations, but to also satisfy performance constraints in the other earth system components including sea ice extent and thickness, ocean north Atlantic Meridional Overturning Circulation (AMOC), and Amazon precipitation. Indeed, it's worth noting that a good deal of effort in tuning went not into tuning the atmosphere itself, but into tuning the whole of the coupled system and remedying emergent coupled climate problems. Indeed we will see that the optimal configuration for CESM2 is not the optimal configuration for CAM6 and WACCM6.

The analysis in this paper strives to address performance in the climate simulation when placed against the backdrop of previous model versions, parameterization selection, and parameter senitiivities. Section \ref{sec:description} outlines the major parameterization changes, with a comparison of simulated climate in \ref{sec:climate}. A more in depth analysis of CLUBB is provided in \ref{sec:tendencies} and an explorations of many of the important parameter and parameterization sensitivities in CAM6 is shown in \ref{sec:sensitivities}. Finally, a brief summary of the major modes of variability is given in \ref{sec:variability} with summary and conclusion in \ref{sec:conclusions}.


Tuning (coupled constarints)
WACCM constraints