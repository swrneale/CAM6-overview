\section{Introduction}
\label{sec:intro}

The National Center for Atmospheric Research (NCAR) Community Atmosphere Model, version 6 (CAM6) is the atmospheric component of the Community Earth System Model, version 2 \cite[CESM2,][]{Danabasoglu2020}. CAM6 was a collaborative development effort between NCAR, partner universities and national labs and is a significant enhancement to the representation of atmospheric processes compared to CAM5 \citep[][]{Neale2012}. Its development targeted existing missing and incomplete processes, and in particular those key to cloud feedbacks \citep{Webb2017}, cloud aerosol interactions \citep{Stevens2015} and climate sensitivity \citep{Zelinka2020}. Consequently cloud processes were the primary focus of the development effort. Historically, parameterizations of cloud processes have been compartmentalized into different approximations of closely related phenomena, namely shallow convection, boundary layer processes and resolved-scale condensation, each employing different approximations. However, this is artificial and is subject to non-commutative process ordering properties \citep{Donahue2018}, in addition to poor scale-aware properties. In CAM6 a partial unification has been implemented with the Cloud Layers Unified By Binormals \citep{Golaz2002,Golaz2002a} scheme, under the umbrella paradigm of high-order moist turbulence. Microphysics, surface stress, vertically propagating gravity waves and aerosol parameterizations further advance the suite of physical processes from CAM5. The combination of these processes have combined to produce climate forcings, and feedbacks significantly different to CAM5 \citep{Gettelman2019}. Understanding these interactions is crucial, given the extensive contributions from CESM2 and CAM6 to the current Coupled Model Intercomparison Project \citep[CMIP6, ][]{Eyring2016a}.

The purpose of this paper is three fold. To document the scientific configuration of CAM6 and analyze changes in the climate features compared to previous model versions. The representation of coupled dynamical large-scale modes of variability \citep{Simpson2020}, and versions of the model with an elevated model top \cite{Gettelman2019} are mostly favorable compared to CAM5. By analyzing AMIP configurations of CAM5 and additionally CAM4 \citep{Neale2013} here, the focus will be on any monotonicity in the recent trajectory of model skill. Meaning, can improvements during this cycle of development be identified as a trend or as a recovery of degradation seen in the preceding model cycle. 

To summarize the regional equilibrium of parameterized processes. In particular, given the more general moniker of moist turbulence in CAM6, is it possible to diagnose the climate regimes in a similar to CAM5 where state tendencies are assigned explicitly to its compartmentalized parameterizations. To attribute net climate differences in CAM6 to incremental physics changes applied as part of the development process. Given the significant modification to cloud feedbacks seen in CAM6 \citep{Gettelman2019} and the resultant increase in the benchmark equilibrium climate sensitivity, from 4.2K to 5.3K, in CESM2 with largely the same ocean component as CESM1 \citep{Bacmeister2020}, it is critical to understand these changes. Importantly, the physics changes are not uniquely selections of parameterization choices between CAM5 and CAM6, but also tuning parameters which frequently have impacts as large as an individual parameterization choice. Unlike previous model versions CESM2 was developed mostly in a fully coupled framework. Indeed, it's worth noting that a good deal of effort in tuning went not into tuning the atmosphere itself, but into tuning the whole of the coupled system. Because of this approach the true sensitivities of the CAM6 climate, to parameterization choice and settings are not well understood in an AMIP framework, and so this analysis aims to understand these further.


The analysis in this paper strives to address performance in the climate simulation when placed against the backdrop of previous model versions, parameterization selection, and parameter sensitivities. Section \ref{sec:description} outlines the major parameterization changes, with a comparison of simulated climate in Section \ref{sec:climate}. A more in depth analysis of CLUBB is provided in Section \ref{sec:tendencies} and an explorations of many of the important parameter and parameterization sensitivities in CAM6 is shown in Section \ref{sec:sensitivities}. Finally, summary and conclusion in Section \ref{sec:conclusions}.

