\section{Process Tendencies}
\label{sec:tendencies}

The changes in the simulated climate are undoubtedly related to the introduction of new physical parameterizations. Identifying their individual roles is a challenge, when analyzing states changes alone. With such a significant change in the paradigm of how atmospheric processes are represented (i.e., with the full moist turbulence in CLUBB), an analysis of the tendencies of prognostic variables, e.g., temperature (dry static energy), momentum, water (vapor,liquid,ice) should illuminate the model behavior. In this section we focus on the temperature and water vapor tendencies to contrast the interaction of processes in CAM5 and CAM6. Figure \ref{f_vtend_DQDT_CAM6} shows the dominant humidity forcing of the boundary layer in CAM6. The sources and sinks of moisture show a dominant source in the sub-tropics and sink in the deep tropics. In CAM6 this is a superposition of the




Sep 25, 2020
* List of figs. to show
-CAM5 lower trop Q
-CAM6 lower trop Q
-Regions: NITCZ DJF, E Pac JJA, SOcn JJA, Strop SOcn, WPac, SAmer, WPac, WIOc
-Yes: N,. ITCZ (DJF), E. Pac JJA, SOcn JJA yes, WIO JJA yes


Since CLUBB is a both a new parameterization and conceptually a new approach to determined moist turbulence in climate models we analyse the its response in CAM6 when compared to the more conventional separation of phenomena-based parameterizations in CAM5. 

Plots
-800-1050 CAM5/CAM6 DQDT Annual
-Profiles
--JJA
-USA, ??
-ITCZ: Definitely JJA  DQDT as it has good PBL top and ZM bulge higher up in CAM5
-WIO ??
-EPAC: Definitely JJA 
-SOCN: DQDT interesting due to microp.
-
-Central US: Temp
-DJF N ITCZ may be good also
-Tibet: JJA DTDT seems most interesting


Could locate in the main bias regions.


What to include
JJA/DJF Prect
JJA/DJF SWCF


*Figs
-2D
-1D regional (WPac/EPac/

*Notes
MG1 has a much greater role in storm track cooling in the lower atmos., similarly for moistening
CLUBB-like tendencies in lower troposphere are large than in CLUBB, similarly for moistening tendencies
MG1 Microphysics drying much greater over W. Pacific
CLUBB-like U-trop heating stronger than CLUBB in Pacific storm track and over Tibet
ZM drying in U-trop much larger in Indo-Pacific in CAM5

