\section{Process Tendencies}
\label{sec:tendencies}

Since CLUBB is a both a new parameterization and conceptually a new approach to determined moist turbulence in climate models we analyse the its response in CAM6 when compared to the more conventional separation of phenomena-based parameterizations in CAM5. 

Plots
-800-1050 CAM5/CAM6 DQDT Annual
-Profiles
--JJA
-USA, ??
-ITCZ: Definitely JJA  DQDT as it has good PBL top and ZM bulge higher up in CAM5
-WIO ??
-EPAC: Definitely JJA 
-SOCN: DQDT interteseting due to microp.
-
-Central US: Temp
-DJF N ITCZ may be good also
-Tibet: JJA DTDT seems most interesting


Could locate in the main bias regions.


What to include
JJA/DJF Prect
JJA/DJF SWCF


*Figs
-2D
-1D regional (WPac/EPac/

*Notes
MG1 has a much greater role in storm track cooling in the lower atmos., similarly for moistening
CLUBB-like tendencies in lower troposphere are large than in CLUBB, similarly for moistening tendencies
MG1 Microphysics drying much greater over W. Pacific
CLUBB-like U-trop heating stronger than CLUBB in Pacific storm track and over Tibet
ZM drying in U-trop much larger in Indo-Pacific in CAM5

